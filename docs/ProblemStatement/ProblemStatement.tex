\documentclass{article}

\usepackage{tabularx}
\usepackage{booktabs}
\usepackage{amssymb}

\title{CAS 741: Problem Statement\\EEG Source Localizer}

\author{Leila Mousapour [mousapol]}

\date{}

%% Comments

\usepackage{color}

\newif\ifcomments\commentstrue

\ifcomments
\newcommand{\authornote}[3]{\textcolor{#1}{[#3 ---#2]}}
\newcommand{\todo}[1]{\textcolor{red}{[TODO: #1]}}
\else
\newcommand{\authornote}[3]{}
\newcommand{\todo}[1]{}
\fi

\newcommand{\wss}[1]{\authornote{blue}{SS}{#1}} 
\newcommand{\plt}[1]{\authornote{magenta}{TPLT}{#1}} %For explanation of the template
\newcommand{\an}[1]{\authornote{cyan}{Author}{#1}}


\begin{document}

\maketitle

\begin{table}[hp]
\caption{Revision History} \label{TblRevisionHistory}
\begin{tabularx}{\textwidth}{llX}
\toprule
\textbf{Date} & \textbf{Developer(s)} & \textbf{Change}\\
\midrule
26/09/2020 & Leila Mousapour & First version of the problem statement was written\\
\bottomrule
\end{tabularx}
\end{table}

\textbf{Input}: Electroencephalogram signals
\\
{\footnotesize 64 channels x time samples (recording time/sampling frequency) $\mu$volt (EEG time series)}
\\*

\textbf{Output}: Active sources inside the brain
\\
{\footnotesize Number of sources x time samples (recording time/sampling frequency)  $\mu$volt (source time courses)}
\\*

Electroencephalography (EEG), which is a method to record electrical activity of
the brain, has a plethora of applications such as decoding mental imageries used
in brain-computer interfaces. One of the big open problems in EEG signal
processing is finding a good feature space after which we can apply machine
learning and classification methods to the data. The standard in the field is to
start with the electrode space and then extract features such as amplitude or
latencies (time domain features), frequency power spectra (frequency features),
common spatial patterns (spatial features) etc. A novel approach that we would
like to investigate is to first map EEG signals from electrode space into
spatial coordinates of the brain to achieve more useful features.

These techniques are known as source localization algorithms which can be
applied to the signals recorded from the scalp and locate the underlying active
sources generating the activity sensed on the electrodes. expected to increase
the signal to noise ratio. Additionally, mapping the activity from 64-channel
space to fewer sources reduces data dimensionality immensely, helps avoid
overfitting and redundancy, leads to better human interpretations and less
computational cost with the simplification of models. Therefore, this scientific
software will implement several techniques to map EEG signals from electrode
space to source space.

% Comments by the instructor were provided in the following format:

% \wss{comment}

% Comments by the author were provided in the following format:

% \an{comment}

\end{document}
