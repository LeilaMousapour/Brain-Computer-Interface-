\documentclass{article}

\usepackage{tabularx}
\usepackage{booktabs}
\usepackage{amssymb}

\title{CAS 741: Problem Statement\\EEG Source Localizer}

\author{Leila Mousapour [mousapol]}

\date{}

\input{../Comments}

\begin{document}

\maketitle

\begin{table}[hp]
\caption{Revision History} \label{TblRevisionHistory}
\begin{tabularx}{\textwidth}{llX}
\toprule
\textbf{Date} & \textbf{Developer(s)} & \textbf{Change}\\
\midrule
26/09/2020 & Leila Mousapour & First version of the problem statement was written\\
5/10/2020 & Leila Mousapour & Second version of the problem statement was written\\
\bottomrule
\end{tabularx}
\end{table}

\textbf{Inputs}: 
\begin{itemize}
  \item EEG signals
  \\ {\footnotesize (Dim: Channel No. x time samples( recording time/sampling frequency) ) ( $\mu$volt )}
  \item Electrode locations
  \\ {\footnotesize (Dim: Channel No.  x 3 coordinations}
  \item Individual’s MRI (Magnetic resonance imaging)
  \item The method (inverse solution technique) 
\end{itemize}
%Electroencephalogram signals

\textbf{Output}: Activity of all sources inside the brain 
\\
{\footnotesize The activity of each voxel of the brain mesh grid in time (source time-course)
\\
(Dim: Number of sources x time samples (recording time/sampling frequency) ) ($\mu$volt )}
\\*

Electroencephalography (EEG), which is a method to record electrical activity of the brain, has a plethora of 
applications such as decoding mental imageries used in brain-computer interfaces. One of the big open problems 
in EEG signal processing is finding a good feature space after which we can apply machine learning and 
classification methods to the data. The standard in the field is to start with the electrode space and then extract 
features such as amplitude or latencies (time domain features), frequency power spectra (frequency features), 
common spatial patterns (spatial features) etc. A novel approach that we would like to investigate is to first map 
EEG signals from electrode space into spatial coordinates of the brain to achieve more useful features.
\\

These techniques are known as  ``source localization `` algorithms which can be applied to the signals recorded 
from the scalp and locate the underlying active sources generating the activity sensed on the electrodes expected 
to increase the signal to noise ratio. Additionally, mapping the activity from an n-channel space to fewer sources 
reduces data dimensionality immensely, helps avoid overfitting and redundancy, leads to better human 
interpretations and less computational cost with the simplification of models. Therefore, this scientific software will 
implement several techniques to map EEG signals from electrode space to source space.


%Comments by the instructor were provided in the following format:
%
%\wss{comment}
%
%Comments by the author were provided in the following format:
%
%\an{comment}

\end{document}
